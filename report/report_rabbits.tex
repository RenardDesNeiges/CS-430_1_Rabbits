\documentclass[11pt]{article}

\usepackage{amsmath}
\usepackage{textcomp}
\usepackage[top=0.8in, bottom=0.8in, left=0.8in, right=0.8in]{geometry}
% Add other packages here %

%%Special equation notation sauce (not sure if I'll need it but it's nice not to have to think about it later)
\usepackage{amssymb}
\usepackage{mathtools}

%% declaring abs so that it works nicely
\DeclarePairedDelimiter\abs{\lvert}{\rvert}%
\DeclarePairedDelimiter\norm{\lVert}{\rVert}%

% Swap the definition of \abs* and \norm*, so that \abs
% and \norm resizes the size of the brackets, and the 
% starred version does not.
\makeatletter
\let\oldabs\abs
\def\abs{\@ifstar{\oldabs}{\oldabs*}}
%
\let\oldnorm\norm
\def\norm{\@ifstar{\oldnorm}{\oldnorm*}}
\makeatother



% Put your group number and names in the author field %
\title{\bf Exercise 1.\\ Implementing a first Application in RePast: A Rabbits Grass Simulation.}
\author{Group \textnumero: 272257, 262609}

\begin{document}
\maketitle

\section{Implementation}

%This section describes the main assumptions that were made in order to implement the simulation described on the Moodle.

\subsection{Assumptions}
% Describe the assumptions of your world model and implementation (e.g. is the grass amount bounded in each cell) %

\subsubsection{Assumptions for the Implementation of grass and of it's growth}

There can be either $1$ or $0$ unit of grass per cell. At each simulation tick \textit{GrassGrowthRate} (default $50$) units of grass are added to empty cells \textbf{if possible}. This value is user defined and modifiable throughout the simulation.

\subsubsection{Assumptions for the Implementation of the movement of Rabbits and of collisions}

At each tick each alive rabbit tries to move to a random cell picked among the 4 cells adjacent to it's location. If it tries to move to a cell where another rabbit is present, \textbf{if the cell is empty it moves to it, if it is already occupied it doesn't move for this turn}. Regardless of whether it actually moved or not the rabbit loses $1$ energy unit per tick.

\subsubsection{Assumptions for the Implementation of feeding, energy and reproduction}

Each rabbit has an energy value $e \in [0,20]$ this value drops by a unit of $1$ every turn. At each tick, rabbits "eat" grass after moving (in the new cell they moved to if they succeeded in moving). If there exists a unit of grass on the cell it's on the energy of the rabbit "eats" it (clearing the cell of grass) and the rabbit's energy increases by a value \textit{GrassEnergy} (default $5$) that is user-set and editable throughout the simulation. At each tick (before movement and feeding) if a rabbit has an energy value $e \geq \text{ \textit{BirthThreshold}}$ it gives birth (this means if a rabbit has gathered enough energy through eating it will give birth at the start of the next turn). \textit{BirthThreshold} (default $20$) is user set and editable throughout the simulation. 

\subsubsection{Assumptions for the Implementation of birth and death}

When giving birth, \textbf{the parent rabbit gives a random proportion of it's energy to it's child}, meaning it losses some random amount of it's energy. The child inherits the exact amount of energy the parent lost when giving birth and appears on a random empty cell on the grid (the cell can have grass but no rabbit). The repartition of energy between parent and child is the following : $e_{parent} + e_{child} = e_{\text{parent before birth}}$. At each tick, after movement, feeding, and reproduction rabbits whose energy is $\leq 0$, die: they are removed from the simulation.

\subsubsection{Assumptions for the initialization of the sim}

The model is initialized with a user-set \textit{NumInitGrass} (default $100$) number of grass cells (if this amounts can possibly fit on the grid, if it cannot then there may be less grass in the simulation). A user set \textit{NumInitRabbits} (default $30$) defines the initial number of rabbits in a similar way to the grass. Initial rabbits are given a random amount of energy $e \in [0,20]$.

\subsection{Implementation Remarks}
% Provide important details about your implementation, such as handling of boundary conditions %

In our model implementation energy is stored as a \textit{double} value. The way the placement of new grass units and rabbits is handled is by picking a random position repeatedly until an empty cell is found, this operation can be repeated a finite number of times to avoid infinite loops in the case where no cell is free. This solution runs the risk of not finding a free spot in edge cases when there may actually be one but has the advantage of being relatively computationally inexpensive (in the most common case) compared to going through every single free cell every time. An alternative solution could be the storage of a separate list of every free cell but such an implementation would be quite memory-intensive for large size simulations.

%%talk about rabbit shuffling

%%energy is a double value

%%management of occupied cells and threshold

\section{Results}
% In this section, you study and describe how different variables (e.g. birth threshold, grass growth rate etc.) or combinations of variables influence the results. Different experiments with different settings are described below with your observations and analysis

\subsection{Experiment 1}

\subsubsection{Setting}

\subsubsection{Observations}
% Elaborate on the observed results %

\subsection{Experiment 2}

\subsubsection{Setting}

\subsubsection{Observations}
% Elaborate on the observed results %

\vdots

\subsection{Experiment n}

\subsubsection{Setting}

\subsubsection{Observations}
% Elaborate on the observed results %

\end{document}